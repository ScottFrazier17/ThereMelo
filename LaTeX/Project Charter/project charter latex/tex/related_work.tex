
In the realm of hand-tracking technology, the "state-of-the-art" has seen significant advancements in recent years, offering various solutions across different domains. These solutions vary from academic research prototypes to commercially available products. In the context of your project, which aims to create a hand-tracking program for music generation similar to playing the theremin using the Leap Motion camera and advanced software, it's essential to survey the existing landscape and understand why these solutions may not fully meet your customer's needs.

The landscape of hand-tracking technology has undergone rapid and extensive transformations in recent times. Notably, there are innovative devices that employ "free-space, unencumbered hand-tracking using an array of distance sensors" \cite{Chronopoulos2021}, alongside various other cutting-edge techniques. In our pursuit of advancing ThereMelo, we aim to seamlessly integrate these evolving techniques, unlocking new possibilities and applications to take the hand-tracking experience to the next level.

Hand-tracking technology has found diverse applications in today's world, with one prominent domain being the medical industry. An illustrative case in point is its application in "surgical telestration" \cite{Muller2022}, where surgeons can benefit from precise hand-tracking for enhanced surgical procedures. However, despite its widespread utility, hand-tracking hasn't seen as much adoption in the realm of music and entertainment. This is where ThereMelo comes into play. Our mission is to harness the power of ThereMelo to bridge this gap and bring the captivating world of music interaction to the general public, making it more accessible and engaging.

Devices such as the Leap Motion camera \cite{Vysocky2020} are pioneers in hand-tracking technology. Their device, the Leap Motion Controller, offers precise hand and finger tracking. However, while it provides high-quality data, it might not be optimized for the specific musical interaction you're envisioning. It primarily focuses on gesture recognition rather than music generation. This is precisely why we aim to use their newest technology, Leap Motion 2, and ThereMelo in collaboration to help bridge this gap.

There are current devices that, similar to ThereMelo, will take in motion-tracking data and process it in order to manipulate music. "Users can easily
control music equipment and achieve high accuracy of music control information." \cite{Pan2021} But the use cases for these current systems are used for things like pause/play functions. The ThereMelo will take these functions and expand on their usefulness to include things like changing pitch, volume, and notes.

Included in this application of hand-tracking to music is the "use of a gesture-based inter-action game as an entertainment system but, also as a hand-tracking evaluation tool." \cite{Figueiredo2012}. These systems are not currently tailored to the creation and expressive freedoms that are allowed by the use of physical instruments. This is the main purpose for our advancement of ThereMelo. We expect this new technology to not only make the music and entertainment industry more accessible to those who seek it but also to make the interaction between humans and music more engaging.

Limitations that might exist for the existing solutions can include:

1.) Costs- The actual commercial theremins that exist in the market can be very expensive for many users. Music and theremin enthusiasts might be looking out for an alternate affordable option.


